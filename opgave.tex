\section{Indledning} 

Verden er mere og tættere forbundet end nogensinde; men også mere
polariseret.  Vi befinder os i en brydningstid, hvor der i
stigende grad savnes samfundsmæssig konsensus om en 'facit'
omkring, hvordan verden arter sig og vi skal agere i den. Vi
samles omkring forskellige konstellationer af verdensopfattelser,
Positioner, og dermed også identitet, forhandles i sociale
sammenhænge - også SoMe
for at opnå en fornemmelse af tilhørighed blandt
meningsfæller((KILDE)). 

Et særligt tilfælde af en moderne moralsk forbrydelse, er når
synden 'blot' er, at gengive sit indsocialiserede verdensbillede.
Opleves der, at nogen siger dette verdensbillede i mod, kan dette
kan være meget svært at acceptere. Man oplever, at ens
selvopfattelse og selvbillede krænkes, og tager i forargelse til
tasterne for, at tale sin sag op. Vandene går hurtigt højt, idet
man søger at forsvare sin egen (gruppes) position. Sammenholdt med
‘the online disinhibition effect' ((KILDE)), hvor almindelig
person + publikum + anonymitet = grimme ord, skal det ikke meget
til, før miljøet online bliver meget giftigt.

\subsection{Fokusområde}

Et specifikt eksempel på dette generelle fænomen så vi den 13.
januar 2019.  Gillette offentliggjorde en reklame((HENVISNING)) på
YouTube , der i det første minut udlægger eksempler på, det værste
af mandlig opførsel ((NORMATIVT?/VÆRDILADET?)).  Mænd lægger ord i
kvinders mund, står og kigger på drenge der sloges ('det er blot
drengestrege'), opfører sig seksuelt upassende overfor kvinder,
mobber og driller.

Indtil der sker noget. I hvert tilfælde er der nogen der tager
affære - stopper en kammerat fra at fløjte efter en pige; griber
ind i slåskampen; jager mobberene væk.  Altid er der unge drenge
der kigger på.

Budskabet virker svært at tage fejl af. Dagens unge drenge kigger
på os mænd, for at lære hvordan de skal opføre sig når de engang
bliver mænd selv.  Dette expliciteres i videoen -\textit{The boys
of today will be the men of tomorrow}.

Med andre ord: Vi socialiseres (blandt andet) efter eksempler. Ved
at justere vores eksempler, kan vi også justere
socialiseringsudfald. Ligefrem og ligetil.

Jævnfør mit blik på forargelse over oplevet krænkelse af
selvopfattelse, kom der en kraftig modreaktion fra (primært) mænd.
De så ikke så den pågældende adfærd som et 'problem' der skulle
'løses', endsige noget at undskylde og beklage for. Gillette blev
beskyldt for, at ville udslette mænd og mandighed, og der blev
opfordret til, at boycotte Gillettes skrabere og barberblade. 

Der kom (forventeligt) en reaktion på modreaktionen; hvor det, at
ikke kunne genkende denne form for maskulinitet som noget
'forkert' blev fremstillet som bevis for nødvendigheden af dette
budskab. 

Disse reaktioner udspillede sig online, ligesom videoen. Twitter,
Instagram og Facebook var de primære slagmarker i kampen for
definitionsret over begrebet 'maskulinitet'.

\subsection{Problemformulering}

Som ridset op ovenfor, viser reaktionerne på reklamen forskellige
perspektiver og forståelser af begreberne 'mandighed' og
'maskulinitet'. Disse perspektiver ser jeg som ydre tegn på
socialiseringprocesser og socialiseringsudfald. De kommer til
udtryk online, både som led i identitetsdannelse og
selviscenesettelse, såvel som positioneringsprocesser, både
indenfor en given tilhørighedsgruppe og disse grupper i mellem.

Næsten upåagtet hvilket felt man går ind i, er der flere
positioner der kæmper om legitimitet.  Vi har alle normative
forventinger baseret på egne erfaringer og holdninger - man
forventer, at andre ligner sig selv indtil andet er bevist
((KILDE)). Når dette viser sig at ikke være tilfældet, vil der, i
et givet fællesskab, opstå positioneringskampe ((KILDE)). 

Jeg vil se på sociale medier som socialiseringsarena, udtrykt ved
positionineringer omkring maskulinitet på Instagram i kølvandet af
reklamen Gillette offentliggjorde 19. januar 2019.


\section{Forskningsdesign}

Mit oveordnede fokus vil være på sociale medier som
mikrosociologisk arena. Konkret vil jeg tage udgangspunkt i
Instagram, da dette sociale medie har en iboende perfomativ
dimension i udvælgelse af motiv mv.  ((UDDYBES)).  Der er tale om
et medie, hvor man både kan nå ud til flere med sit budskab, og
kan møde meningsfæller i et større omfang end for blot 10 år
siden.  Derudover er der også større mulighed for omhyggelig
iscenesættelse af sig selv, modsat interaktioner i 'den virkelige
verden'.

Jeg vil tage for mig et polariserende emne, og udvœlge eksempler, der hver
illustrerer en legitim, hhv. illegitim, position i de forskellige
konstellationer omkring emnet.   For at kortlægge positioner
omkring emnet? vil jeg lave klyngeanalyser?  netværksanalyser? af
centrale hashtags omkring emnet på instagram (twitter?  facebook).

Jeg vil sammenholde en interaktionsanalyse/tekstanalyse med afsæt
i Goffman's 'Forms of Talk', med en positionsteoretisk tilgang til
emnet. 

\subsection{Sociale medier, socialisering og maskulinitet} 

Der er især socialpsykologien((KILDE?)) der har beskæftiget sig
med sociale mediers betydning for hvordan vi opfører os overfor
hindanden. En af de mest fremtrædende her, er ((HAM MED THE ONLINE
DISINHIBITION EFFECT)). 

Billed-deling som perfomance

NOGLE DER HAR ARBEJDET MED IDENTITETSDANNELSE??

MASKULINITET??

NETVÆRK- positioner

\subsection{Positioneringsteori og interaktionsanalyse}

Positioneringsteori....

Interaktionsanalyser ift goffman.....

\section{Cases}
Jeg vil trække to eksempler på reaktioner ud ((BEGGE KRITISKE
OVERFOR GILLETTE?  ELLER TO MODSATRETTEDE?), og analysere disse i
et positioneringsteoretisk perspektiv.  Hvilke positioner antages
der? Hvilke positioner anerkendes i deres pågældende miljø?



Overordnet kan reaktionerne på Gillette-reklamen deles op i to
grupperinger: dem der er kritiske overfor budskabet; og dem der er
enige i det. ((KILDER))

Fælles for de kritiske er....

De enige mener overvejende at...

Der var også reaktioner — og kommentarer på reaktionerne — i de
mere etablerede mediekanaler. I Teen Vogue kunne man læse, at
selve problemet er, at mænds følelsesmæssige register kun tillader
vrede som reaktion — frem for fx skam, sorg, etc. Flere af
metakommentarerne fortsatte i dette spor, hvor der understreges,
at de voldsomme reaktioner fremhæver nødvendigheden af en
offentlig debat omkring maskulinitet.  

Mit første eksempel er fra motherofsnot på Instagram, der befinder
sig i 'hvad siger i, kan det klæde mænd at være ordentlige?'
lejren.

—-----

THESE ER GOOD VENDINGS MAYBE JEG KAN BRUG DEM IN A LATER AFSNIT

Videnshul: en del skriverier i hvordan disse reaktioner er
symptomer på problemet i vores syn på maskulinitet. Dog er mit
fokus ikke fænomenet manbabies og broflakes specifikt; men i
stedet hvordan den specifikke hændelse har påvirket følelser af
krænkelse og forargelse i de forskellige lejre; og hvilke
mekanismer og signaler der positioneres ud fra.

Litteratur-review?

Hashtag-analyser fra insta: kan se hvilke tags der følges ad;
FB/twitter har likes/retweets/shares, der giver flere indikationer
på 'engagement'

Der opstiles et kontra-argument (stråmand/falsk ækvivalens) i
begrebet 'giftig feminitet'

Positioner, og dermed også identitet, forhandles i sociale
sammenhænge - også SoMe


