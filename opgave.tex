\section{Problemstilling}

Jeg vil undersøge forargelse som fænomen, med sociale medier som
identitets- og socialiseringsarena.

Verden er mere og tættere forbundet end nogensinde; men også mere
polariseret. Man kan danne interessefællesskaber som aldrig før, men man
kan også samle sig omkring forskellige konstellationer af
verdensopfattelser, hvor det kan være meget svært at acceptere, hvis nogen
siger dette verdensbillede i mod. Inde i en given filterboble bliver det
taget for givet, at vacciner skaber autisme; eller at mælk/gluten er
sundhedsskadeligt; eller at feminazierne er skyld i, at mænd ikke må være
maskuline længere; eller at der er (eller ikke er) strukturelle og
sociokulturelle forhold, der stadig spænder ben for
ligestillingsprojektet; eller at muslimske indvandrere er ved at formere
sig til herredømme i Europa. Bliver der stillet spørgsmål mod dette, går
vandene hurtigt højt, idet man søger at forsvare sin (åbenlyst korrekte,
forstås) position. Sammenholdt med ‘the greater internet fuckwad theory’,
hvor almindelig person + publikum + anonymitet = grimme ord, er miljøet
online hurtigt meget giftigt.

Næsten upåagtet hvilket felt man går ind i, er der flere positioner der
kæmper om legitimitet. Selv indenfor den 'almindelige' socialisering, er
vi i en brydningstid ((KILDE)), hvor a) der er ikke en samfundsmæssig
facit længere, og b) der er altid nogen klar til, at påpege hvilke fejl du
nu har begået. Mulighederne for at begå utilgivelige moralske synder er
større end aldrig før. 

Et særligt tilfælde af en moderne moralsk forbrydelse, er når synden
'blot' er, at gengive sit indsocialiserede verdensbillede. Vi står nu alle
[os vestlige, hvide imperialister] til ansvar for fordums generationers
udøvelse af symbolsk vold. Hvis man ikke kan genkende sig dette, eller,
næsten endnu værre, ikke ønsker at forsone (offentligt) på vegne af (egen
eller fælles) fortid, risikerer man, at blive dømt ude af en række
fællesskaber der bryster sig af sit fokus på inklusivitet. Den hvide (mands)
byrde rekontekstualiseres. Det er ikke pålagt et frygteligt ansvar i at
civilisere de uciviliserede; nu er det netop denne holdning der skal gøres op
med.

Internettet bugner efterhånden af hvide riddere, der styrter ind
i diskussioner for at kæmpe for og forsvare grupper de ikke selv er del af, og
som ikke (nødvendigvis) har bedt om hjælp. Jævnfør internettets
dishibitionerende effekt, som nævnt ovenfor, er der hurtigt mange der kan få
næsen i en bjørnefælde af meget grimme kommentarer.

\section{Problemformulering}

Forargelse opstår, når man oplever, andre opføre sig moralsk forkasteligt, uden
at udvise anger. ((KILDE)) Vi har alle normative forventinger baseret på egne
erfaringer og holdninger - man forventer, at andre ligner sig selv indtil andet
er bevist ((KILDE)).

Når dette viser sig at ikke være tilfældet, vil der, i et givet
fællesskab, opstå positioneringskampe ((KILDE)). 

I forlængelse af ovenstående vil jeg undersøge hvordan man vinder eller taber
moralsk anseelse ved iscenesættelse af sig selv på sociale medier.

Hvordan reagerer man på andres forargelse over egne handlinger, når man ikke
selv oplever disse handlinger som moralsk forkastelige? Hvilken normativ
forståelse af moralsk korrekt adfærd vinder frem? Og hvordan opnås legitimitet
til, at påberåbe sig moralsk [overlegenhed|korrekthed]?

\section{Forskningsdesign}

Mit oveordnede fokus vil være på sociale medier som mikrosociologisk
arena. Konkret vil jeg tage udgangspunkt i Instagram, da dette sociale medie
her en iboende perfomativ dimension i udvælgelse af motiv mv.  ((UDDYBES)).
Der er tale om et medie, hvor man både kan nå ud til flere med sit budskab, og
kan møde meningsfæller i et større omfang end for blot 10 år siden. Derudover
er der også større mulighed for omhyggelig iscenesættelse af sig selv, modsat
interaktioner i 'den virkelige verden'.

Jeg vil tage for mig et polariserende emne, og udvœlge eksempler, der hver
illustrerer en legitim, hhv. illegitim, position i de forskellige
konstellationer omkring emnet.

For at kortlægge positioner omkring emnet? vil jeg lave klyngeanalyser?
netværksanalyser? af centrale hashtags omkring emnet på instagram (twitter?
facebook).

Jeg vil sammenholde en interaktionsanalyse/tekstanalyse med afsæt i Goffman's
'Forms of Talk', med en positionsteoretisk tilgang til emnet.

\subsection{Emne}

Den 13. januar 2019 offentliggjorde Gillette en reklame((HENVISNING)), der på
knappe to minutter udlægger eksempler på, det værste af mandlig opførsel. Mænd
lægger ord i kvinders mund, står og kigger på drenge der sloges ('det er blot
drengestrege'), opfører sig seksuelt upassende overfor kvinder, mobber og
driller.

Indtil der sker noget. I hvert tilfælde er der nogen der tager affære - stopper
en kammerat fra at fløjte efter en pige; griber ind i slåskampen; jager
mobberene væk.  Altid er der unge drenge der kigger på.

Budskabet virker svært at tage fejl af. Dagens unge drenge kigger på os mænd,
for at lære hvordan de skal opføre sig når de engang bliver mænd selv.  Dette
expliciteres i videoen -\textit{The boys of today will be the men of tomorrow}.

Med andre ord: Vi socialiseres (blandt andet) efter eksempler. Ved at justere
vores eksempler, kan vi også justere socialiseringsudfald. Ligefrem og ligetil.

Jævnfør mit blik på forargelse ((OGSÅ NOGET MED KRÆNKELSE AF SELVOPFATTELSE)),
kom der en kraftig modreaktion fra mænd, der ikke så den pågældende adfærd som
et problem der skulle løses, endsige noget at undskylde og beklage for. Der kom
(forventeligt) en reaktion på modreaktionen; hvor det, at ikke kunne genkende
denne form for maskulinitet som noget 'forkert' blev fremstillet som bevis for,
at an behøvede dette budskab.

Gillette blev beskyldt for, at ville udslette mænd og mandighed, og der blev
opfordret til, at boycotte Gillettes skrabere og barberblade.

\subsubsection{Cases}

Jeg vil trække to eksempler på reaktioner ud ((BEGGE KRITISKE OVERFOR GILLETTE?
ELLER TO MODSATRETTEDE?), og analysere disse i et positioneringsteoretisk
perspektiv.  Hvilke positioner antages der? Hvilke positioner anerkendes i
deres pågældende miljø?

Mit første eksempel er fra motherofsnot på Instagram, der befinder sig i 'hvad
siger i, kan det klæde mænd at være ordentlige?' lejren.

—-----

THESE ER GOOD VENDINGS MAYBE JEG KAN BRUG DEM IN A LATER AFSNIT

Videnshul: en del skriverier i hvordan disse reaktioner er symptomer på
problemet i vores syn på maskulinitet. Dog er mit fokus ikke fænomenet
manbabies og broflakes specifikt; men i stedet hvordan den specifikke hændelse
har påvirket følelser af krænkelse og forargelse i de forskellige lejre; og
hvilke mekanismer og signaler der positioneres ud fra.

Litteratur-review?

Hashtag-analyser fra insta: kan se hvilke tags der følges ad; FB/twitter har
likes/retweets/shares, der giver flere indikationer på 'engagement'

Der opstiles et kontra-argument (stråmand/falsk ækvivalens) i begrebet 'giftig
feminitet'

Positioner, og dermed også identitet, forhandles i sociale sammenhænge - også
SoMe


white mans burden rekontekstualiseres - at ikke være bevidst om, at man skal
gøre bod for, og stå til ansvar for, fortidens strukturelle kræfter (i denne
sammenhæng primært overfor kvinder)
