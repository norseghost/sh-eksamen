\section{Problemstilling}

Jeg vil undersøge forargelse som fænomen, med sociale medier som
identitets- og socialiseringsarena.

Verden er mere og tættere forbundet end nogensinde; men også mere
polariseret. Man kan danne interessefællesskaber som aldrig før, men man
kan også samle sig omkring forskellige konstellationer af
verdensopfattelser, hvor det kan være meget svært at acceptere, hvis nogen
siger dette verdensbillede i mod. Inde i en given filterboble bliver det
taget for givet, at vacciner skaber autisme; eller at mælk/gluten er
sundhedsskadeligt; eller at feminazierne er skyld i, at mænd ikke må være
maskuline længere; eller at der er (eller ikke er) strukturelle og
sociokulturelle forhold, der stadig spænder ben for
ligestillingsprojektet; eller at muslimske indvandrere er ved at formere
sig til herredømme i Europa. Bliver der stillet spørgsmål mod dette, går
vandene hurtigt højt, idet man søger at forsvare sin (åbenlyst korrekte,
forstås) position. Sammenholdt med ‘the greater internet fuckwad theory’,
hvor almindelig person + publikum + anonymitet = grimme ord, er miljøet
online hurtigt meget giftigt.

Næsten upåagtet hvilket felt man går ind i, er der flere positioner der
kæmper om legitimitet. Selv indenfor den 'almindelige' socialisering, er
vi i en brydningstid ((KILDE)), hvor a) der er ikke en samfundsmæssig
facit længere, og b) der er altid nogen klar til, at påpege hvilke fejl du
nu har begået. Mulighederne for at begå utilgivelige moralske synder er
større end aldrig før. 

Et særligt tilfælde af en moderne moralsk forbrydelse, er når synden
'blot' er, at gengive sit indsocialiserede verdensbillede. Vi står nu alle
[os vestlige, hvide imperialister] til ansvar for fordums gerenationers
udøvelse af symbolsk vold. Hvis man ikke kan genkende sig dette, eller,
næsten endnu værre, ikke ønsker at forsone (offentligt) på vegne af (egen
eller fælles) fortid, risikerer man, at blive dømt ude af en række
fællesskaber der bryster sig af sit fokus på inklusivitet. Den hvide
(mands) byrde rekontekstualiseres. Det er ikke pålagt et frygteligt ansvar
i at civilisere de uciviliserede; nu er det netop denne holdning der skal
gøres op med.

Internettet bugner efterhånden af hvide riddere, der styrter ind
i diskussioner for at kæmpe for og forsvare grupper de ikke selv er del
af, og som ikke (nødvendigvis) har bedt om hjælp. Jævnfør internettets
dishibitionerende effekt, som nævnt ovenfor, er der hurtigt mange der kan
få næsen i en bjørnefælde af meget grimme kommentarer.
\section{Problemformulering}

Forargelse opstår, når man oplever, andre opføre sig moralsk forkasteligt,
uden at udvise anger. ((KILDE)) Hvordan reagerer man på andres forargelse
over egne handlinger, når man ikke selv oplever disse handlinger som
moralsk forkastelige? Vi har alle normative forventinger til egne
erfaringer og holdninger - man forventer, at andre ligner sig selv indtil
andet er bevist ((KILDE)). Når dette viser sig at ikke være tilfældet, vil
der opstå kampe om legitimitet og positioner i et fællesskab. Jeg vil
undersøge hvordan man tilkendegiver sin position, alternativt taber den,
ved iscenesættelse af sig selv på sociale medier.


\section{Forskningsdesign}

SoMe som mikrosociologisk/interaktionistisk arena
Bredere rœkkevidde, men (mulighed for) mere styring
    - hvad giver kapital? Hvilke roller besættes?
    - tekstanalyse -> goffmann, 'forms of talk'
    - positioneringsteori


- gilette-reklamen, med tilhørende furore
    - mænd, der blev skide vrede
    - andre, der sagde 'jo klart vi skal være ordentlige'
    - Gilette gør det for penge

    Cases: reddit-diskussion, på forskellige subs?


- tykke/ikke modeltynde piger der viser hud offentligt/fatshaming generelt
    - får billeder taget ned hurtigt, også selv om det ikke er usømmeligt
    (udover deres vægt)

- white mans burden rekontekstualiseres
    - at ikke være bevidst om, at man skal gøre bod for, og stå til ansvar
    for, fortidens strukturelle kræfter
    - kulturel appropriation
    
    cases:
    
    Karen Templer, og hendes blogindlæg om at rejse til Indien, der
    (givetvis?) reproducerede hendes indsocialiserede verdensbillede:
    andre lod sig støde/krænke, og stillede hende til (overdrevet stort)
    ansvar

    Lærke, der uskyldigt lagde billede af dreads-extensions oppe, og fik
    rigtig meget lort i hovedet


