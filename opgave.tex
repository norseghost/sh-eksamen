\section{Indledning} 


Verden er mere og tættere forbundet end nogensinde; men også mere
polariseret.  Vi befinder os i en brydningstid, hvor der i
stigende grad savnes samfundsmæssig konsensus om en 'facit'
omkring, hvordan verden arter sig og vi skal agere i den. Vi
samles omkring forskellige konstellationer af verdensopfattelser,
for at opnå en fornemmelse af tilhørighed blandt
meningsfæller((KILDE)). 

Internettet gør det muligt, at finde dette fællesskab og denne
samhørighed på nye måder. Marginaliserede individer kan finde
sammen, upåagtet geografi, og deler frit og åbent erfaringer med
hinanden ((SULER)). Der dannes positioner, hvor rettigheder,
friheder, ansvar og pligter forhandles i en evigt regenererende
diskurs ((POS TEORI BOG)).

Opleves der, at nogen siger det verdensbillede man har
internaliseret i mod, kan dette kan være meget svært at acceptere.
Man oplever, at ens selvopfattelse og selvbillede krænkes, og den
position man har antaget ikke altid bliver anerkendt af andre,
eller at man ikke vil anerkende den position, andre forsøger at
placere en i.

De samme processer der gør, at mennesker kan tillade sig at være
mere sårbar og ærlige online, er dog også til grund for meget af
den grove tone man hurtigt finder online. Man tager i forargelse
til tasterne for, at tale sin sag op.  Vandene går hurtigt højt,
idet man søger at forsvare sin egen (gruppes) position.  Denne
ondartede udgave af online disinhibition betyder, at der ikke skal
meget til, før miljøet online hurtigt bliver giftigt. ((SULER))

Jeg vil undersøge, hvordan disse processer arter sig på sociale
medier. Konkret vil jeg tage Instagram som præsentation- og
positionsarena for mig. 

Det følgende afsnit vil præsentere mit teoretiske grundlag for
denne undersøgelse. Derefter vil jeg introducere mit konkrete
fokusområde, med en efterfølgende kort gennemgang af undersøgelser
med et lignende fokus.

Herefter vil jeg præsentere mit empiriske analysegrundlag, med
tilhørende diskussion af mulige analyseresultater. Afslutningsvis
vil jeg diskutere og perspektivere mine konklusioner i et bredere
sociologisk perspektiv.

\section{Teoretisk analysegrundlag}

Jeg tager en mikrosociologisk tilgang, der trækker kraftigt på
interaktionsanalyser og positioneringsteori. Jeg befinder mig også
i et grænseland tæt på socialpsykiatrien, og er ikke bleg for, at
anvende psykologiske elementer i den grad jeg finder det relevant.

\subsection{Online kommunikation og manglende hæmninger}

En af de mest fremtrædende forskere i sociale mediers betydning
for individer er den amerikanske psykoanalytiker John Suler, der
beskriver en række forklarende årsager til, at man slipper sine
hæmninger online ((KILDE)).

Herunder disassociativ anonymitet - 'jeg' på de sociale medier er
en anden end 'jeg' på arbejde.  Dermed skal jeg-på-arbejdet ikke
stå til ansvar for, hvad jeg-på-instagram foretager mig. Tæt
beslægtet er den disassociative fantasi - den online verden er en
anden verden, uafhængig af den udenfor. Den indirekte, asynkrone
kontakt gør endvidere den anden 'usynlig' for en - man undgår
øjenkontakt, og kan flygte fra åstedet uden at skulle stå til
direkte ansvar for sin udlevering.

Suler skælner mellem en godartet og en ondartet udgave af denne
disinhibitionseffekt. Når man taler om uhensigtsmæssig opførsel
online, er det primært det sidste, der tales om; kendetegnet af
vrede, had, og vold. Den godartede er kendetegnet ved, en form for
berabejdning og øget forståelse af sig selv, ifølge Suler, hvor
den ondartede blot er destruktiv udtømning. For et nyligt fænomen
på godartet online disinhibition, kan man blot se på de sidste års
bekendelser under \#MeToo.

\subsection{Positioneringsteori}

Positioner, og dermed også identitet, forhandles i sociale
sammenhænge - også SoMe

\subsection{Interaktionsanalyser} 

\subsubsection{Selvudstillelse eller selvskabelse?}

Uddybe/perspektivere til instagram - de elementer ift aktiv
konstruktion/selektiv effekt af respons/erkendelse af den
konstruerede natur er absolut relevante

Suler, J: From self portraits to selfies

et selvportræt får andre mennesker til at fremstå mere ægte - og
er dermed en meget effektiv måde, at løfte sløret for sig slev
samtidig med, at man forsøger at styre denne fremvisning af
selvet.

objektive (illusion af fotograf) vs subjektivt (tydeligt brug af
remedier) - subjektive mere almindelige, da de er en del af en
selv-narration af ens pågående livshistorie; og også nemmere at
tage (i udgangspunktet?)

Selfiet/SoMe opslag i almindelighed gør det mere bevidst for den
enkelte, at man aktivt konstruerer sig selv - det observerende ego

Selektiv effekt af respons - søger billeder der får mange
likes/kommentarer; underkender dele af identiet/livsoplevelser;
risiko for, at låse fast i en bestemt selvopfattelse.

Erkendelse af, at det hele er opsat; 'ny' trend, hvor ærlighed,
sårbarhed mv, det ægte anerkendes (men er dog, IMHO, lige så
konstrueret)

\section{Lignende undersøgelser}

- formentlig meget omkring positionering og kønsroller; der er
også skrevet om SoMe ift Goffmann; og en hel del om hvordan
identitet forhandles online.

\subsection{Andre områder der viser de samme tendenser} 

Der var også reaktioner — og kommentarer på reaktionerne — i de
mere etablerede mediekanaler. I Teen Vogue kunne man læse, at
selve problemet er, at mænds følelsesmæssige register kun tillader
vrede som reaktion — frem for fx skam, sorg, etc. Flere af
metakommentarerne fortsatte i dette spor, hvor der understreges,
at de voldsomme reaktioner fremhæver nødvendigheden af en
offentlig debat omkring maskulinitet. 

\subsection{mit videnshul}

en del skriverier i hvordan disse reaktioner er symptomer på
problemet i vores syn på maskulinitet.  Dog er mit fokus ikke
fænomenet manbabies og broflakes specifikt; men i stedet hvordan
den specifikke hændelse har påvirket følelser af krænkelse og
forargelse i de forskellige lejre; og hvilke mekanismer og
signaler der positioneres ud fra.  Et særligt tilfælde af en
moderne moralsk forbrydelse, er når synden 'blot' er, at gengive
sit indsocialiserede verdensbillede.


\section{Fokusområde}

I dagene efter den 13.  januar 2019 kunne man meget tydeligt se
postitioneringsprocesser i praksis på de sociale medier.  Gillette
offentliggjorde en reklame((HENVISNING)) på YouTube , der i det
første minut udlægger eksempler på, en særlig stereotypi af
mandlig opførsel.  Mænd lægger ord i kvinders mund, står og kigger
på drenge der sloges ('det er blot drengestrege'), opfører sig
seksuelt upassende overfor kvinder, mobber og driller.

Indtil der sker noget. I hvert tilfælde er der nogen der tager
affære - stopper en kammerat fra at fløjte efter en pige; griber
ind i slåskampen; jager mobberene væk.  Altid er der unge drenge
der kigger på.

Budskabet virker svært at tage fejl af. Dagens unge drenge kigger
på os mænd, for at lære hvordan de skal opføre sig når de engang
bliver mænd selv.  Dette expliciteres i videoen -\textit{The boys
of today will be the men of tomorrow}.

Med andre ord: Vi socialiseres (blandt andet) efter eksempler. Ved
at justere vores eksempler, kan vi også justere
socialiseringsudfald. Ligefrem og ligetil.

Jævnfør mit blik på forargelse over oplevet krænkelse af
selvopfattelse, kom der en kraftig modreaktion fra (primært) mænd.
De så ikke så den pågældende adfærd som et 'problem' der skulle
'løses', endsige noget at undskylde og beklage for. Gillette blev
beskyldt for, at ville udslette mænd og mandighed, og der blev
opfordret til, at boycotte Gillettes skrabere og barberblade. 

Der kom (forventeligt) en reaktion på modreaktionen; hvor det, at
ikke kunne genkende denne form for maskulinitet som noget
'forkert' blev fremstillet som bevis for nødvendigheden af dette
budskab. 

Disse reaktioner udspillede sig online, ligesom videoen. Twitter,
Instagram og Facebook var de primære slagmarker i denne kamo for
definitionsret over begrebet 'maskulinitet'.

\subsection{Problemformulering}

Som ridset op ovenfor, viser reaktionerne på reklamen forskellige
perspektiver og forståelser af begreberne 'mandighed' og
'maskulinitet'. Disse positioner, om man vil, ser jeg som ydre
tegn på socialiseringprocesser og socialiseringsudfald. 

Jeg vil se på sociale medier som socialiseringsarena, udtrykt ved
positionineringer omkring maskulinitet på Instagram i kølvandet af
reklamen Gillette offentliggjorde 19. januar 2019.
((SOCIALISERING?))


Mit oveordnede fokus vil være på sociale medier som
mikrosociologisk arena. Konkret vil jeg tage udgangspunkt i
Instagram, da dette sociale medie har en iboende perfomativ
dimension i udvælgelse af motiv mv.  ((UDDYBES)).  Der er tale om
et medie, hvor man både kan nå ud til flere med sit budskab, og
kan møde meningsfæller i et større omfang end for blot 10 år
siden.  Derudover er der også større mulighed for omhyggelig
iscenesættelse af sig selv, modsat interaktioner i 'den virkelige
verden'. ((ME LIKE - METAKOMMUNIKATION/OPSUMMERING FTW))

\section{Forskningsdesign} 

Indenfor det åbenlyst polariserende emne beskrevet ovenfor, vil
jeg trække to modsatrettede positioner ud, og analysere disse i et
positioneringsteoretisk perspektiv.  Hvilke positioner antages
der? Hvilke positioner anerkendes i deres pågældende miljø? Hvilke
positioner placerer de selv andre i?

Denne positioneringsteoretiske tilgang vil blive suppleret med
interaktionsanalytiske begreber ((??? VIL JEG NU OGSÅ DET??)).
:w: For at beskrive nogle af positionerne omkring emnet vil jeg
lave ((HVILKE))analyser af centrale hashtags omkring emnet på
Instagram.   (twitter?  facebook). ((ER DER PLADS/TID HERFOR??))


Jeg vil sammenholde en interaktionsanalyse/tekstanalyse med afsæt
i Goffman's 'Forms of Talk', med en positionsteoretisk tilgang til
emnet. 


\section{Empirisk analysegrundlag}

Overordnet kan reaktionerne på Gillette-reklamen deles op i to
grupperinger: dem der er kritiske overfor budskabet; og dem der er
enige i det. ((KILDER))

Fælles for de kritiske er....

De enige mener overvejende at...
 

Mit første eksempel er fra motherofsnot på Instagram, der befinder
sig i 'hvad siger i, kan det klæde mænd at være ordentlige?'
lejren.

—-----

THESE ER GOOD VENDINGS MAYBE JEG KAN BRUG DEM IN A LATER AFSNIT







Litteratur-review?

Hashtag-analyser fra insta: kan se hvilke tags der følges ad;
FB/twitter har likes/retweets/shares, der giver flere indikationer
på 'engagement'

Der opstiles et kontra-argument (stråmand/falsk ækvivalens) i
begrebet 'giftig feminitet'

Positioner, og dermed også identitet, forhandles i sociale
sammenhænge - også SoMe 

Hvordan medieres positionskamperne på SoMe evt af den online
disinhibition?

Beskrive søgestrategi? evt for at understrege videnshul

Opstil argumenter i litteratur som positioner, mit bidrag kan
placere sig i forold til - ikke blit liste uden struktir

Posititionsbeskrivelser!

international litteratur - check! 

opsamling til sidst

Metode efter forskningslandskab og underhypoteser

Ikke lav en historie - beskriv hvad du finder

indledning - lav en kort sammenfatning af opgaven, cf UIP

begrænse offentlig debat -som argument-
