\section{Indledning} 


Verden er mere og tættere forbundet end nogensinde; men også mere
polariseret.  Vi befinder os i en brydningstid, hvor der i
stigende grad savnes samfundsmæssig konsensus om en 'facit'
omkring, hvordan verden arter sig og vi skal agere i den. Vi
samles omkring forskellige konstellationer af verdensopfattelser,
for at opnå en fornemmelse af tilhørighed blandt
meningsfæller \autocite{sulerUniqueGroupsCyberspace1999}.

Internettet gør det muligt, at finde dette fællesskab og denne
samhørighed på nye måder. Marginaliserede individer kan finde
sammen, upåagtet geografi, og deler frit og åbent erfaringer med
hinanden \autocite[s. 184]{sulerOnlineDisinhibitionEffect2004}.  
Der dannes positioner, hvor rettigheder,
friheder, ansvar og pligter forhandles i en evigt regenererende
diskurs \autocite[s. 22]{harreromPositioningTheoryMoral1999}.

Opleves der, at nogen siger det verdensbillede man har
internaliseret i mod, kan dette kan være meget svært at acceptere.
Man oplever, at ens selvopfattelse og selvbillede krænkes. Den
position man har antaget ikke altid bliver anerkendt af andre,
eller at man ikke vil anerkende den position, andre forsøger at
placere en i \autocite[s30]{harreromPositioningTheoryMoral1999}.

De samme processer der gør, at mennesker kan tillade sig at være
mere sårbare og ærlige online, er dog også til grund for meget af
den grove tone man hurtigt finder på de sociale medier. Denne 
ondartede udgave af online disinhibition betyder, at der ikke skal
meget til, før miljøet online hurtigt bliver giftigt
\autocite{sulerOnlineDisinhibitionEffect2004}.

Derudover har internettet bidraget med nye settings for at vise 
hvem man \sout{vil fremstå som at være} er. De sociale medier 
giver os nye rammer for sociale interaktioner, og nye scener for 
fremvisning af de roller vi ønsker at blive identificeret ved.

Jeg vil undersøge, hvordan disse processer arter sig på sociale
medier. Konkret vil jeg tage Instagram som præsentation- og
positionsarena for mig.

Det følgende afsnit vil præsentere mit teoretiske grundlag for
denne undersøgelse. Derefter vil jeg introducere mit konkrete
fokusområde, med en efterfølgende kort gennemgang af undersøgelser
med et lignende fokus.

Herefter vil jeg præsentere mit empiriske analysegrundlag, med
tilhørende diskussion af mulige analyseresultater. Afslutningsvis
vil jeg diskutere og perspektivere mine konklusioner i et bredere
sociologisk perspektiv.

\section{Teoretisk analysegrundlag}

Jeg tager en mikrosociologisk tilgang, der trækker kraftigt på
interaktionsanalyser og positioneringsteori. Jeg befinder mig også
i et grænseland tæt på socialpsykiatrien, og er ikke bleg for, at
anvende psykologiske elementer i den grad jeg finder det relevant.

\subsection{Online kommunikation og manglende hæmninger}

En af de mest fremtrædende forskere i sociale mediers betydning
for individer er den amerikanske psykoanalytiker John Suler, der
beskriver en række forklarende årsager til, at man slipper sine
hæmninger online \autocite{sulerOnlineDisinhibitionEffect2004}.

Her vil jeg fremhæve \emph{disassociativ anonymitet} - 'jeg' på de 
sociale medier er en anden end 'jeg' i den analoge verden.  Dermed
skal analoge!jeg ikke stå til ansvar for, hvad online!jeg 
foretager mig. Tæt beslægtet er den disassociative fantasi - den
online verden er en anden verden, uafhængig af den udenfor. Den
indirekte, asynkrone kontakt gør endvidere den anden 'usynlig' for
en - man undgår øjenkontakt, og kan flygte fra åstedet uden at
skulle stå til direkte ansvar for sin udlevering.

Suler skelner mellem en godartet og en ondartet udgave af denne
disinhibitionseffekt. Når man taler om uhensigtsmæssig opførsel
online, er det primært det sidste, der tales om; kendetegnet af
vrede, had, og vold. Den godartede er kendetegnet ved, en form for
bearbejdning og øget forståelse af sig selv, ifølge Suler, hvor
den ondartede blot er destruktiv udtømning. For et nyligt eksempel
på godartet online disinhibition, kan man se på det sidste års
fænomen \#MeToo.

\subsection{Positioneringsteori}

\citeauthor{harrePositioningTheoryMoral1999} beskriver 
positioneringsteori som (\citeyear[s.  1, min oversættelse
]{harrePositioningTheoryMoral1999}):
\begin{quotation}
    \ldots studiet af lokale morale ordener som evigt skiftende 
    mønstre af gensidige og omtvistelige rettigheder og 
    obligationer af tale og handlen \ldots
\end{quotation}

Positioner påvirker dermed hvordan og hvorledes individuelle 
handlemåder og handlemuligheder. Er man positioneret som uvidende 
om et emne, tildeles man ikke ret til at bidrage til diskussioner 
herom. Derudover er positioner generelt set relationelle, idet at 
nogen må indtage positionen “magtesløs” for at andre kan have 
positionen “mægtig”.

Positioneringsteori vil undersøge, hvordan diskursive processer 
producerer psykologiske fænomener. Den tager udgangspunkt i, at 
vores oplevede liv brydes op til \emph{episoder} i gennem 
diskurser. Det er disse episoder, der er grundlaget for både vores 
livshistorier og den sociale verden \autocite[s.  
4]{harrePositioningTheoryMoral1999}.

Episoder i denne kontekst omfatter “enhver serie hændelser hvor 
mennesker deltager, hvor der er en form for enhedsprincip”. De 
indbefatter, ud over synlig adfærd, også deltagernes tanker, 
følelser, intentioner med mere. De både former deltagernes 
handlinger, og er definerede af dem.

Hvor en positioneringsteoretisk analyse skiller sig fra for 
eksempel en interaktionsanalyse på baggrund af 
\citeauthor{goffmanPresentationSelfEveryday1956} i
\citetitle{goffmanPresentationSelfEveryday1956}, er et explicit 
fokus på situationens historicitet. Episoder bliver til på 
baggrund af foregående episoder, og vil  indeholde elementer der 
ikke kan forklares på baggrund af generelle regler og roller 
\autocite[s. 5-6]{harrePositioningTheoryMoral1999}.

Dermed fremhæves tre grundlæggende egenskaber af, hvordan sociale 
og psykiske fænomener konstrueres 
\autocite{harrePositioningTheoryMoral1999}:
\begin{enumerate}
    \item
        deltagernes morale positioner, og tilhørende ret og pligt 
        til bestemte ytringer
    \item
        samtalens historik
    \item
        de faktiske udsagn, med kraft til at give form til dele af 
        den sociale verden
\end{enumerate}

% FIXME: Jarring transition
Når positionshandlinger skal klassificeres, præsenterer 
\citeauthor{harrePositioningTheoryMoral1999} tre niveauer:
\begin{enumerate}
    \item
        positioneres individer af individer og kollektiver af 
        kollektiver?
    \item
        positionerer individet/kollektivet refleksivt sig selv, 
        eller er det ved en anden, der positionerer og bliver 
        positioneret?
    \item
        Er positioneringshandlingen symmetrisk eller asymmetrisk?
\end{enumerate}

\subsubsection{Positioneringens forståelsesramme}

Indenfor positioneringsteorien anses mennesker for at være 
placeringer for sociale handlinger, hvor 'samtalen' er den 
grundlæggende bestanddel af den sociale verden. Der er i samtalen,
den sociale verden bliver skabt; og sociale handlinger genereres 
og reproduceres \autocite[s. 
15]{harrePositioningTheoryMoral1999}.

I denne forståelse af den sociale verden som bestående af 
\emph{personer} og \emph{samtaler}, er positionering en “\ldots 
diskursiv konstruktion af personlige fortællinger, der gør en 
persons handlinger forståelige og relativt fastgjorte som sociale 
handlinger, og hvori samtalens medlemmer har specifikke 
placeringer.” \autocite[s. 16]{harrePositioningTheoryMoral1999}

Dermed positioner i en samtale en samling af personens moralske og 
personlige egenskaber. Man kan positionere sig — eller blive 
positioneret — som mægtig eller magtesløs, selvsikker eller 
undskyldende, med videre. En handlings sociale gennemslagskraft og 
deltagernes positioner er videre forbundne, idet samtaler har 
historier, og de positioner der indtages i en samtale er forbundne 
med disse historier. Der er dermed en gensdigt skabende triade, 
bestående af følgende elementer \autocite[s.  
17-18]{harrePositioningTheoryMoral1999}:

% TODO: make pretty diagram
position -- social kraft af -- historie

% FIXME: very terse
Positioner kan opstå “naturligt” i situationen; men en besiddelse 
af den dominerende rolle i en samtale vil kunne tvinge de andre 
talere i positioner de ellers ikke ville  have indgået frivilligt.  
Dog kan disse begyndende positioner anfægtes, og talerne dermed 
blive ompositionerede. Ved at starte en samtale af en højere 
orden, hvor den foregående samtale blot er et emne, kan man 
positionere sig som kommentator på positioner, historier og 
sociale handlinger deri.

\subsection{Selvets præsentation}

\subsubsection{Selvudstillelse eller selvskabelse?}

Uddybe/perspektivere til instagram - de elementer ift aktiv
konstruktion/selektiv effekt af respons/erkendelse af den
konstruerede natur er absolut relevante

Suler, J: From self portraits to selfies

et selvportræt får andre mennesker til at fremstå mere ægte - og
er dermed en meget effektiv måde, at løfte sløret for sig slev
samtidig med, at man forsøger at styre denne fremvisning af
selvet.

objektive (illusion af fotograf) vs subjektivt (tydeligt brug af
remedier) - subjektive mere almindelige, da de er en del af en
selv-narration af ens pågående livshistorie; og også nemmere at
tage (i udgangspunktet?)

Selfiet/SoMe opslag i almindelighed gør det mere bevidst for den
enkelte, at man aktivt konstruerer sig selv - det observerende ego

Selektiv effekt af respons - søger billeder der får mange
likes/kommentarer; underkender dele af identiet/livsoplevelser;
risiko for, at låse fast i en bestemt selvopfattelse.

Erkendelse af, at det hele er opsat; 'ny' trend, hvor ærlighed,
sårbarhed mv, det ægte anerkendes (men er dog, IMHO, lige så
konstrueret)

\section{I videnskabelig dialog}

- skal proquest søgestreng medtages? Skal i hvertfald beskrive at 
jeg søger på...

- formentlig meget omkring positionering og kønsroller; der er
også skrevet om SoMe ift Goffmann; og en hel del om hvordan
identitet forhandles online.

\subsection{Andre områder der viser de samme tendenser} 

Der var også reaktioner — og kommentarer på reaktionerne — i de
mere etablerede mediekanaler. I Teen Vogue kunne man læse, at
selve problemet er, at mænds følelsesmæssige register kun tillader
vrede som reaktion — frem for fx skam, sorg, etc. Flere af
metakommentarerne fortsatte i dette spor, hvor der understreges,
at de voldsomme reaktioner fremhæver nødvendigheden af en
offentlig debat omkring maskulinitet. 

\subsection{mit videnshul}

en del skriverier i hvordan disse reaktioner er symptomer på
problemet i vores syn på maskulinitet.  Dog er mit fokus ikke
fænomenet manbabies og broflakes specifikt; men i stedet hvordan
den specifikke hændelse har påvirket følelser af krænkelse og
forargelse i de forskellige lejre; og hvilke mekanismer og
signaler der positioneres ud fra.  Et særligt tilfælde af en
moderne moralsk forbrydelse, er når synden 'blot' er, at gengive
sit indsocialiserede verdensbillede.


\section{Fokusområde}

I dagene efter den 13.  januar 2019 kunne man meget tydeligt se
postitioneringsprocesser i praksis på de sociale medier.  Gillette
offentliggjorde en reklame((HENVISNING)) på YouTube , der i det
første minut udlægger eksempler på, en særlig stereotypi af
mandlig opførsel.  Mænd lægger ord i kvinders mund, står og kigger
på drenge der sloges ('det er blot drengestrege'), opfører sig
seksuelt upassende overfor kvinder, mobber og driller.

Indtil der sker noget. I hvert tilfælde er der nogen der tager
affære - stopper en kammerat fra at fløjte efter en pige; griber
ind i slåskampen; jager mobberene væk.  Altid er der unge drenge
der kigger på.

Budskabet virker svært at tage fejl af. Dagens unge drenge kigger
på os mænd, for at lære hvordan de skal opføre sig når de engang
bliver mænd selv.  Dette expliciteres i videoen -\textit{The boys
of today will be the men of tomorrow}.

Med andre ord: Vi socialiseres (blandt andet) efter eksempler. Ved
at justere vores eksempler, kan vi også justere
socialiseringsudfald. Ligefrem og ligetil.

Jævnfør mit blik på forargelse over oplevet krænkelse af
selvopfattelse, kom der en kraftig modreaktion fra (primært) mænd.
De så ikke så den pågældende adfærd som et 'problem' der skulle
'løses', endsige noget at undskylde og beklage for. Gillette blev
beskyldt for, at ville udslette mænd og mandighed, og der blev
opfordret til, at boycotte Gillettes skrabere og barberblade. 

Der kom (forventeligt) en reaktion på modreaktionen; hvor det, at
ikke kunne genkende denne form for maskulinitet som noget
'forkert' blev fremstillet som bevis for nødvendigheden af dette
budskab. 

Disse reaktioner udspillede sig online, ligesom videoen. Twitter,
Instagram og Facebook var de primære slagmarker i denne kamo for
definitionsret over begrebet 'maskulinitet'.

\subsection{Problemformulering}

Som ridset op ovenfor, viser reaktionerne på reklamen forskellige
perspektiver og forståelser af begreberne 'mandighed' og
'maskulinitet'. Disse positioner, om man vil, ser jeg som ydre
tegn på socialiseringprocesser og socialiseringsudfald. 

Jeg vil se på sociale medier som socialiseringsarena, udtrykt ved
positionineringer omkring maskulinitet på Instagram i kølvandet af
reklamen Gillette offentliggjorde 19. januar 2019.
((SOCIALISERING?))


Mit oveordnede fokus vil være på sociale medier som
mikrosociologisk arena. Konkret vil jeg tage udgangspunkt i
Instagram, da dette sociale medie har en iboende perfomativ
dimension i udvælgelse af motiv mv.  ((UDDYBES)).  Der er tale om
et medie, hvor man både kan nå ud til flere med sit budskab, og
kan møde meningsfæller i et større omfang end for blot 10 år
siden.  Derudover er der også større mulighed for omhyggelig
iscenesættelse af sig selv, modsat interaktioner i 'den virkelige
verden'. ((ME LIKE - METAKOMMUNIKATION/OPSUMMERING FTW))

\section{Forskningsdesign} 

Indenfor det åbenlyst polariserende emne beskrevet ovenfor, vil
jeg trække to modsatrettede positioner ud, og analysere disse i et
positioneringsteoretisk perspektiv.  Hvilke positioner antages
der? Hvilke positioner anerkendes i deres pågældende miljø? Hvilke
positioner placerer de selv andre i?

Denne positioneringsteoretiske tilgang vil blive suppleret med
interaktionsanalytiske begreber ((??? VIL JEG NU OGSÅ DET??)).
:w: For at beskrive nogle af positionerne omkring emnet vil jeg
lave ((HVILKE))analyser af centrale hashtags omkring emnet på
Instagram.   (twitter?  facebook). ((ER DER PLADS/TID HERFOR??))


Jeg vil sammenholde en interaktionsanalyse/tekstanalyse med afsæt
i Goffman's 'Forms of Talk', med en positionsteoretisk tilgang til
emnet. 


\section{Empirisk analysegrundlag}

For at kortlægge de forskellige positioner omkring denne 
diskussion af maskulinitet starter jeg med at lave en optælling af 
hashtags på Instagram.

inden jeg tæller op, foretager jeg dog nogle databehandlingsgreb:
- jeg laver all tekst til små bokstaver
- jeg slår \emph{gilette} og \emph{gillette} sammen
- jeg begrænser mig til opslag lavet efter reklamen blev
offentliggjort ((indtil hvornår?))


Overordnet kan reaktionerne på Gillette-reklamen deles op i to
grupperinger: dem der er kritiske overfor budskabet; og dem der er
enige i det. ((KILDER))

Fælles for de kritiske er....

De enige mener overvejende at...

Hold øje på de forskellige niveauer i positionering omkrng emnet

reklamen
individuelle positioner


Mit første eksempel er fra motherofsnot på Instagram, der befinder
sig i 'hvad siger i, kan det klæde mænd at være ordentlige?'
lejren.

—-----

THESE ER GOOD VENDINGS MAYBE JEG KAN BRUG DEM IN A LATER AFSNIT







Litteratur-review?

Hashtag-analyser fra insta: kan se hvilke tags der følges ad;
FB/twitter har likes/retweets/shares, der giver flere indikationer
på 'engagement'

Der opstiles et kontra-argument (stråmand/falsk ækvivalens) i
begrebet 'giftig feminitet'

Positioner, og dermed også identitet, forhandles i sociale
sammenhænge - også SoMe 

Hvordan medieres positionskamperne på SoMe evt af den online
disinhibition?

Beskrive søgestrategi? evt for at understrege videnshul

Opstil argumenter i litteratur som positioner, mit bidrag kan
placere sig i forold til - ikke blit liste uden struktir

Posititionsbeskrivelser!

international litteratur - check! 

opsamling til sidst

Metode efter forskningslandskab og underhypoteser

Ikke lav en historie - beskriv hvad du finder

indledning - lav en kort sammenfatning af opgaven, cf UIP

begrænse offentlig debat -som argument-
